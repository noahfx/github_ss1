\documentclass{beamer}
\usepackage[utf8]{inputenc}
\usepackage[spanish]{babel}
\usepackage{hyperref}
\usepackage{ulem}\normalem
\usepackage{graphicx}
\parskip 10.9pt
\usetheme{Darmstadt}
\usecolortheme {default}
\title{Github}
\author{Josu\'e Ortega }
\institute{Seminario de Sistemas 1}
\begin{document}
\begin{frame}
\titlepage
\end{frame}
\section{Github -- Social Coding(for all)}
\begin{frame}
\begin{itemize}
\item http://github.com
\item Servicio web de alojamiento para desarrollo de software.
\item Utiliza el controlador de versiones GIT
\end{itemize}
\end{frame}
\begin{frame}{Tipos de servicio}
\begin{itemize}
\item {\em Pagado:} Proyectos privados 
\item {\em Gratuito:} El c\'odigo que se aloja es Open Source.
\end{itemize}
\end{frame}

\section {C\'odigo Social}
\begin{frame}{C\'odigo Social}
El sitio provee la funcionalidad de una red social como:
\begin{itemize}
\item feeds.
\item Seguidores.
\item Graficas para seguir como los developers trabajan en su servidor de 
  versiones.
\end{itemize}
\end{frame}
\section{Trabajo en equipo}
\begin{frame}{Trabajo en equipo}
Github provee la facilidad de hacer pull requests y pedir permisos para
commits al dueño del proyecto. Tambien la posibilidad de seguir la actividad
en los proyectos de interes para el desarrollador.
\end{frame}
\section{Requerimientos}
\begin{frame}{Requerimientos}
Se necesita del controlador de versiones git(http://en.wikipedia.org/wiki/Git\_(software)).
\\
Si el usuario lo decide, puede usar su llave publica para la autenticaci\'on con 
el servidor remoto para mayor seguridad. 
\end{frame}
\end{document}
